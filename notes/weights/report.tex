\documentclass[fleqn,oneside,12pt]{article}
\date{}
\newcommand{\up}[1]{a_{#1}^{\dagger}}
\newcommand{\down}[1]{a_{#1}}
\newcommand{\ud}{\mathrm{d}}
\newcommand{\EE}{\mathrm{E}}
\newcommand{\diff}[2]{\frac{\ud {#1}}{\ud {#2}}}
\newcommand{\pdiff}[2]{\frac{\partial #1}{\partial #2}}
\newcommand{\fl}{\mathrm{fl}}
\newcommand{\sqrtfrac}[2]{\sqrt{\frac{#1}{#2}}}
\newcommand{\ui}{\mathbf{i}}
\newcommand{\uj}{\mathbf{j}}
\newcommand{\ua}{\mathbf{a}}
\newcommand{\ub}{\mathbf{b}}
\newcommand{\bi}{\bar{i}}
\newcommand{\bj}{\bar{j}}
\newcommand{\ba}{\bar{a}}
\newcommand{\bb}{\bar{b}}
\newcommand{\tb}{\phantom{}^{Q} \hat b}
\newcommand{\erf}{\mathrm{erf}}
\newcommand{\erfc}{\mathrm{erfc}}
\setlength{\parskip}{1ex plus 0.5ex minus 0.2ex}
\setlength{\parindent}{0pt}
\usepackage[top=1in, bottom=1in, left=1in, right=1in]{geometry}
\usepackage{amsmath}
\usepackage[super]{cite}
\usepackage{amssymb}
\usepackage[usenames,dvipsnames]{color}
\usepackage{verbatim}
\usepackage{fancyhdr}
\usepackage{graphicx}
\pagestyle{fancyplain}
\begin{document}
\lhead{Cheesy Time-Dependent Properties in AIMS}
\rhead{Rob Parrish}

\textbf{Definitions:} The AIMS wavefunction is,
\[
| \Psi (x, t) \rangle
\equiv
\sum_{I}
c_{I} (t)
| \chi_{I} (x, t) \rangle
| I \rangle
\]
where $c_{I} (t)$ is the TBF amplitude, $| \chi_{I} (x,t) \rangle$ is a frozen
Gaussian nuclear basis function, and $| I \rangle$ is the adiabatic electronic
state.

\textbf{Approximation 1:} We wish to compute an arbitrary time-dependent
observable depending on the nuclear coordinates,
\[
\bar O (t)
\equiv
\langle \Psi (x, t) |
O (x)
| \Psi (x, t) \rangle
\]
Plugging in, this is,
\[
\bar O(t)
=
\sum_{IJ}
c_{I}^{*}
c_{J}
\delta_{IJ}^{\mathrm{e}}
\langle \chi_{I} (x, t) |
O (x)
| \chi_{J} (x, t) \rangle
\]
I think an OK approximation is,
\[
\bar O(t)
\approx
\sum_{IJ}
c_{I}^{*} (t)
S_{IJ} (t)
c_{J} (t)
O (\bar x_{IJ} (t)) 
\]
Here,
\[
\bar x_{IJ} (t)
\equiv
\frac{\langle \chi_{I} (x, t) | x | \chi_{J} (x, t) \rangle}
{\langle \chi_{I} (x, t) | \chi_{J} (x, t) \rangle}
=
\frac{\alpha_{I} x_{I} (t) + \alpha_{J} x_{J} (t)}{\alpha_{I} + \alpha_{J}}
=
\frac{1}{2}
\left [
\bar x_{I} (t)
+
\bar x_{J} (t)
\right ]
\]
Here the last equality holds only for $\alpha_{I} = \alpha_{J}$ (common in
AIMS).

The approximation invoked above is,
\[
\langle \chi_{I} (x, t) | O (x) | \chi_{J} (x, t) \rangle
\approx
\langle \chi_{I} (x, t) | O (\bar x_{IJ}) | \chi_{J} (x, t) \rangle
=
S_{IJ}
O(\bar x_{IJ})
\]
This will be accurate if $O(x)$ varies slowly from $O(\bar x_{IJ})$ relative to
the integration weight $\chi_{I}^{*} (x, t) \chi_{J} (x, t)$. In AIMS, the TBFs
are quite narrow, so this is probably a fine approximation.

\textbf{Approximation 2:} Another possible approximation is,
\[
\bar O(t)
\approx
\sum_{IJ}
c_{I}^{*} (t)
S_{IJ} (t)
c_{J} (t)
\frac{1}{2}
\left [
O(\bar x_{I} (t))
+
O(\bar x_{J} (t))
\right ]
\]
\[
=
\sum_{I}
O (\bar x_I (t))
\sum_{J}
\frac{1}{2}
\left [
c_{I}^{*} (t)
S_{IJ} (t)
c_{J} (t) 
+
c_{J}^{*} (t)
S_{JI} (t) 
c_{I} (t)
\right ]
\]
\[
\equiv
\sum_{I}
O (\bar x_I (t))
q_{I} (t)
\]
This is the Mulliken-charge-based estimate, or as Todd prefers, the ``bra-ket
averaged Taylor approximation''

\end{document}
